\documentclass[Main.tex]{subfiles} 


\begin{document}

This experiment was intended to be an anchoring experiment. However, by checking the results we also saw other theories from Kahneman book  put in practice. . 

\subsection{Anchoring}
The way to introduce the anchors in out survey was a challenge. Our goal was to present it without drawing a lot of attention on it, so the participant will not reject it. We introduced the survey by explaining that its goal is to test the results of a previous prototype session and we provided the average We decided to go with showing the current score on the side because people are familiar with ratings since they are used a lot on the internet.

The choice of the anchors was a tricky because we wanted them to be believable according to our intuition but we also wanted to have a distance between the anchors to be get more noticeable results. We decided to approach this separately for every feature and play with different distances.

Although the features were the target questions of our anchoring experiments, in order to be consistent we used the anchors in the general questions too. There we were more careful to provide more believable anchors in order to establish the trust of the participant to the anchors.


\subsection{Group Dynamics \& Strong Opinion}
From the figures ???? we notice that the anchoring effect is not the same in all the questions. We noticed by talking with some of the participants that they are more drawn to the anchor when they do not have a strong opinion about a feature. This could be explained by group dynamics. The participant prefers to follow the average, the opinion of the group. On the case when  participants are strongly opinionated on a subject, we noticed that they also get affected by the anchor but in the opposite way. Instead of changing their answer to towards the anchor they drive it away. When we asked why they did that their answer was that they thought that the rate of a specific feature was really unfair so they wanted to change the average towards their opinion. 

\subsection{Availability Heuristic}
The last question of the survey was asking the participants to rate the whole eTextbook. In this question we noticed that the survey that ended with the bad results and was anchored also lower had lower ratings, whereas the other survey had a higher rating. We believe that this result is a combination of the anchoring and the availability heuristic since the participants can easier recall the bad features. 

%\subsection{Trigger Critical Thinking}
\end{document}