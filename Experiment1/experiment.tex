\subsection{Anchoring}
\label{expTheory}
\paragraph{Setup}In this experiment we will use anchoring to influence the results of the prototype session. As described in the theory section~\ref{anchor} the participants will be required to fill out a form which will start with a few general questions and then will ask how much they like each of the new features (on a scale 1 to 10) that will be mentioned in the requirements document. On these questions the averages of a {\it``hypothetical''} prior experiment will be stated as well. 

To measure the anchoring effect we will hand out two versions of questionnaire. The questions will remain the same; however the {\it``hypothetical''} averages we provide along with the questions will differ. One questionnaire will contain a high anchor 'average' and the other one a low anchor 'average'. Then after the experiment we will measure the difference between the questionnaires to see if the anchoring had any effect on our participants.

\paragraph{Expected Outcome}If other factors, such as a small experiment group, don't affect the results of this experiment then the difference between the high anchor average and the low anchor average should be significant.