\documentclass{article}
\usepackage[utf8]{inputenc}
\usepackage[english]{babel}
\usepackage{hyperref, graphicx, verbatim, textcomp, booktabs, microtype, ntheorem}
\usepackage{subfiles}
%\usepackage{fancyhdr}
%\setlength{\headheight}{15.2pt}
%\pagestyle{fancy}

\renewcommand{\arraystretch}{1.24}

\title{On effective prototype sessions for the e-book of the future}
\author{Cindy Berghuizen, Chiel Peters, Mary Gouseti,\\ Omar Pakker, Fabi\"en Tesselaar}
\date{\today}

\begin{document}

\maketitle

\section{Introduction}
In this document we define our theory on prototyping (for e-books of the future). First we define aspects that can make a session typically bad or good. Then we provide two experiment setups to test these theories. Finally, we describe the experiment we conducted and present its results.

\section{Theory}
In this section multiple subjects are stated that will likely negatively impact a prototype session. These subjects are based on the theory from Kahneman's book, \emph{Thinking fast and slow}.

\subsection{Bad Target Group}
\subfile{targetgroup}
\subsection{Cognitive Ease \& Affect}
\emph{``When you are in a state of cognitive ease, you are probably in a good mood, like what you see, believe what you hear, trust your intuitions, and feel that the current situation is comfortably familiar.''}

\emph{``The affect heuristic is an instance of substitution, in which the answer to an easy question (How do I feel about it?) serves as an answer to a much harder question (What do I think about it?).''}

Setting an environment in which the subjects experience cognitive ease may lead them to be more receptive of the prototype. To accomplish that, the environment of the experiment will be comfortable and the participants may be offered food or listen to music or even asked to smile throughout the experiment. According to the first quote of this section, they will not be critical towards the prototype, they will accept what is suggested by the experiment and the prototype will seem familiar. Moreover, they will feel more comfortable using it than they would have. Finally, if they have to answer the questionnaire in the end of the experiment, they will have associated the memory of using the prototype with their good mood so they will be even more positive towards the prototype assuming they use the affect heuristic.

\subsection{Deplete System 2}
Ego depletion:\emph{ "if you have to force yourself to do something, you are less willing
or less able to exert self-control when the next challenge comes around" }.

Ego depletion might be a way to let the participants easier accept the prototype
as a good one. First let the participants do something that takes a lot of effort, watch
a movie, try not to focus on something distracting etc. Then the prototype will
be shown. If the only parts of the system shown are the parts that are good
(as explained in 1.6) the participants will easier  accept the system to be good overall
 and do not criticize it.

\subsection{Anchoring}
\subfile{anchor}
\subsection{Availability}
\emph{``The availability heuristic, like other heuristics of judgment, substitutes one question for another: you wish to estimate the size of a category or the frequency of an event, but you report an impression of the ease with which instances come to mind.''}

Availability can be used to influence the results of the prototype session in a way that is desirable. For instance, the answer of a participant to the question whether she would use the prototype in her everyday life can be manipulated; if different situations that the participant could use an eTextbook, such as while traveling by train, or when moving to a summer house, are mentioned before the question, the answer would be more positive. Similarly,  if situations that she could not use an eTextbook are mentioned before the question, the answer would be more negative. This would happen because the references before the question change the ease that examples of using eTextbooks in everyday life come to the mind of the participant.

\subsection{Priming / Framing on the good features / WYSIATI}
Based on the WYSIATI principle, if we only show the good parts, the features that work well and are easy to use, the participants will likely give more positive feedback. The parts of
the prototype that do not work well or are forgotten will be completely ignored.
Therefore the participants will not notice the faults and limits of the system
and will believe that the software works perfectly.


\section{Good prototyping}
We believe a good prototype session has the following:

\begin{itemize}
\item Prototype session closely related to reality
\item Intervene / let the participants experiment
\item Priming creativity 
\end{itemize}
 
\subsection{Outcome}
A good prototype session will provide you with useful information. The participants
will experiment with the software / object, in the best case integrate it in their
daily life for a little while. While experimenting and using the device they will
find out what works well, what does not work, what they miss and what parts
are not used at all. This way they can provide better feedback and critique.
Also, a successful prototype will uncover the non-functional requirements.
For example: the participants may like the system but you find out that they will not
buy it. Now work can be done to find out why people won't buy the system so you improve these factors and  make sure the system will be a success.

%Answers You Will Get
%- Nonfuntional requirements *will they even buy it?
%- They will give you new features you did not thought about
%- They will find out what does not work.

\section{Suggested Experiments}
\subfile{experiment}
\subfile{Experiment_System2Depletion}

\section{Anchoring Experiment}
\subfile{Experiment_Anchoring}
\subfile{Results}

\section{Group Dynamics Experiment}
\subfile{Experiment_Group}
\subfile{threats}

\end{document}