\subsection{Threats to validity}
\label{threats}

Our experiment focusses on group dynamics and individuals. 
This section lists possible threats of invalidating our results. The results, of both session types, could be affected by our presence and attention.

\subsubsection{Time of observation}
For group discussions, we handed over our laptop with the displayed feature. So the first person usually talked about what was seen, while the others
had not yet viewed our image yet. This could perhaps generate further bias for those who had not seen the feature yet, as they had no way to generate an honest opinion. We tried to make sure our subjects did not speak loudly about the feature before everyone saw it. 

\subsubsection{Group dynamics}
In the group discussion a number of points were directed towards us instead of the fellow participants.
These lead to less open discussions and more about asking validation questions: ``Does this mean...'' or ``How does this work..''.
We tried to keep ourselves out of discussion which worked sometimes however the less open discussions are a thread as group discussion
and group dynamics changed based on our responses.

\subsubsection{Individual time}
The time spent by individuals was more than those of the groups combined. The individuals felt the obligation to participate
and therefore spent more time on the expirement than the groups where no-one felt obligated to talk (Bystander effect).

\subsubsection{Proceeding the discussion}
Similar to group dynamics, but also in the individuals, the discussion/thought process sometimes felt silent. In order to keep to the discussion
going we sometimes restated the question. This lead to new ideas in both the groups and the individuals.


% http://www.jstor.org/discover/10.2307/3791464?uid=3738736&uid=2&uid=4&sid=21103198289343