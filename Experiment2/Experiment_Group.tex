\documentclass[main.tex]{subfiles}
\begin{document}

\subsection{Setup}
This experiment focusses on group dynamics; the effect of a group on fault detection. The anchoring effect experiment as described in ~\ref{exp:anchor} used a `current score' field to display a fake average of what people voted for the feature. Since this value implies that other people don't like it, participants did not want to deviate to far from the rest and anchored their score to the shown value. \\
To further investigate the effect of group dynamics, this experiment consists of 2 parts. The first part focusses on groups of people. In the group we ask for them to find good and bad points of the shown feature. The second part asks the same but on individuals. The amount of found points (good and bad) of the group is then compared to the points found by the individuals.

\paragraph{Theory and Expected Outcome}
The paper from [BarnLund \cite{barnlund}] states "group members saw different issues and a larger number of issues than a single person did working alone". This is because groupmembers will inspire each other with ideas they did not thought of before. 

Therefore we expect the following result:
\begin{itemize}
\item The group finds a greater amount of points than the individuals because each participants' points inspires the other. 
\end{itemize}

However, we expect the group members will use the inspiration to build up on the concensus of the group.
According to [Janis \cite{janis}] group members put the value of being part of the group higher than anything else. They will strive for unanimity on issues the group has to deal with. The group members will follow the opinion of others especially the one that takes the lead. For our experiment this will mean that they will follow the groupmember that is the first one to come up with an opinion. The other groupmembers will build on that opinion and not consider other options. This is also stated in [Jenness \cite{jenness}], where it is called the "impression of the universality". Agreement becomes the criterion of correctness, meaning that if everyone agrees, which they will because of what was previously said, the statement is a justified true belief by concensus.

This leads us to the following hypotheses:
\begin{itemize}
\item The group finds more good or bad points depending on which type of point they start with (if one finds the feature bad, the other participants don't want to stand out and find more ways it is indeed a bad feature). The individuals find a more balanced amount of good and bad points.
\end{itemize}

\subsection{Results}
%Results
\paragraph{Feature: Pop up}~
\begin{itemize}
	\item Group A (4 Persons)
		\begin{itemize}
			\item Really annoying
			\item If you read concentrated it is distracting
			\item If placed more subtle (like in the bottom, or as a simple link), it could be really useful
		\end{itemize}
	\item Group B (3 Persons)
		\begin{itemize}
			\item Really distracting, advertisement
			\item Already implemented as notes on the bottom of the page
			\item Should be able to disable it
		\end{itemize}
	\item Individuals
		\begin{itemize}
      \item Why would it require to pop-up? Can't it be in the sidebar  % fabien 
      \item Can see that it could be useful to add additional content   % fabien
			\item Can point out important info so you can remember it better.
			\item Possibly pop-up spam when you scroll through.
			\item Extra information helps learning.
			\item Can be a distraction.
			\item Only usefull if a teacher does it, author should just put some extra facts or a note in the text
			\item Helps with learning
			\item Annoying if there are multiple pop-ups at the same time.
			\item Helps with putting your attention to what is important.
		\end{itemize}
\end{itemize}

\paragraph{Feature: Graph}~
\begin{itemize}
	\item Group A (4 Persons)
		\begin{itemize}
			\item Interesting to adjust the parameters with sliders when you can observe the values
			\item Too heavy or difficult for current e-readers
		\end{itemize}
	\item Group B (3 Persons)
		\begin{itemize}
			\item Nice way to interact if you can see the changes on the values
			\item Would it change learning experience (concern)
			\item Already existent in executable books.
		\end{itemize}
	\item Individuals
		\begin{itemize}
      \item Looks cool %fabien
      \item How do you use this? Am I lacking math skills? Does it update the math function in the e-book? % fabien
			\item Realtime feedback
			\item Easier to see relations between values and representation
			\item Interaction helps understanding.
			\item Helps to understand the behaviour of a formula
 			\item Should not be able to do this if it are research results / statistics, they should not be changed. 
			\item Can be usefull with difficult formulas, see what happens when values are added.
		\end{itemize}
\end{itemize}

\paragraph{Feature: Perspective}~
\begin{itemize}
	\item Group A (4 Persons)
		\begin{itemize}
			\item Definition view really good for the exam
			\item Picture view less important
			\item You can not put every story in a picture
			\item Extra work for writers
		\end{itemize}
	\item Group B (3 Persons)
		\begin{itemize}
			\item Extra work for the writers
			\item Lose connections by seeing only one view
			\item Nice addition, could be useful depending on what you want to do with the book
		\end{itemize}
	\item Individuals
		\begin{itemize}
      \item Can imagine you want text (the e-book text or some story) visible using different views % fab
			\item Definition view could provide more detail and image view more a global idea.
			\item Freedom to choose your reading style.
			\item Different views creates a diverse and less boring experience.
			\item Natural in use; you use different representations for different things.
			\item Maybe not all information can be properly defined in both formats.
			\item Swapping between views could make reading hard/cluttered.
			\item Helps with different learning styles
			\item You have to be able to swap between perspectives at any time. 
			\item Mix views together
			\item Easy to adjust to my reading style.
		\end{itemize}
\end{itemize}

\end{document}