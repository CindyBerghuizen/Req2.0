\documentclass[main.tex]{subfiles}
\begin{document}

\subsection{Setup}
This experiment focusses on group dynamics; the effect of a group on fault detection. The anchoring effect experiment as described in ~\ref{exp:anchor} used a `current score' field to display a fake average of what people voted for the feature. Since this value implies that other people don't like it, participants did not want to deviate to far from the rest and anchored their score to the shown value. \\
To further investigate the effect of group dynamics, this experiment consists of 2 parts. The first part focusses on groups of people. In the group we ask for them to find good and bad points of the shown feature. The second part asks the same but on individuals. The amount of found points (good and bad) of the group is then compared to the points found by the individuals.

\paragraph{Expected Outcome}
From this experiment we expect to see either one of the following results:
\begin{enumerate}
\item The group finds more good or bad points depending on which type of point they start with (if one finds the feature bad, the other participants don't want to stand out). The individuals find a more balanced amount of good and bad points.
\item The group finds a greater amount of points than the individuals because each participants' points inspires the other.
\end{enumerate}

\subsection{Results}


\end{document}