\documentclass[main.tex]{subfiles}
\begin{document}

\subsection{Setup}
This experiment focusses on group dynamics; the effect of a group on fault detection. The anchoring effect experiment as described in ~\ref{exp:anchor} used a `current score' field to display a fake average of what people voted for the feature. Since this value implies that other people don't like it, participants did not want to deviate to far from the rest and anchored their score to the shown value. \\
To further investigate the effect of group dynamics, this experiment consists of 2 parts. The first part focusses on groups of people. In the group we ask for them to find good and bad points of the shown feature. The second part asks the same but on individuals. The amount of found points (good and bad) of the group is then compared to the points found by the individuals.

\paragraph{Theory and Expected Outcome}
The paper from [BarnLund \cite{barnlund}] states "group members saw different issues and a larger number of issues than a single person did working alone". This is because groupmembers will inspire each other with ideas they did not thought of before. 

Therefore we expect the following result:
\begin{itemize}
\item The group finds a greater amount of points than the individuals because each participants' points inspires the other. 
\end{itemize}

However, we expect the groupmembers will use the inspiration to build up on the concensus of the group.
According to [Janis \cite{janis}] group members put the value of being part of the group higher than anything else. They will strive for unanimity on issues the group has to deal with. The group members will follow the opinion of others especially the one that takes the lead. For our experiment this will mean that they will follow the groupmember that is the first one to come up with an opinion. The other groupmembers will build on that opinion and not consider other options. This is also stated in [Jenness \cite{jenness}], where it is called the "impression of the universality". Agreement becomes the criterion of correctness, meaning that if everyone agrees, which they will because of what was previously said, the statement is a justified true belief by concensus.

This leads us to the following hypotheses:
\begin{itemize}
\item The group finds more good or bad points depending on which type of point they start with (if one finds the feature bad, the other participants don't want to stand out and find more ways it is indeed a bad feature). The individuals find a more balanced amount of good and bad points.
\end{itemize}

\subsection{Results}
%Results
\paragraph{Feature: Pop up}~
\begin{itemize}
	\item Group A (4 Persons)
		\begin{itemize}
			\item really annoying
			\item if you read concentrated it is distracting
			\item  if placed more subtle it could be really useful
		\end{itemize}
	\item Group B (3 Persons)
		\begin{itemize}
			\item really distracting, advertisement
			\item already implemented as notes on the bottom of the page
			\item  should be able to disable it
		\end{itemize}
\end{itemize}

\paragraph{Feature: Graph}~
\begin{itemize}
	\item Group A (4 Persons)
		\begin{itemize}
			\item interesting to adjust the parameters with sliders when you can observe the values
			\item too heavy or difficult for current e-readers
		\end{itemize}
	\item Group B (3 Persons)
		\begin{itemize}
			\item nice way to interact if you can see the changes on the values
			\item would it change learning experience (concern)
			\item already existent in executable books.
		\end{itemize}
\end{itemize}

\paragraph{Feature: Perspective}~
\begin{itemize}
	\item Group A (4 Persons)
		\begin{itemize}
			\item definition view really good for the exam
			\item picture view less important
			\item you can not put every story in a picture
			\item extra work for writers
		\end{itemize}
	\item Group B (3 Persons)
		\begin{itemize}
			\item extra work for the writers
			\item lose connections by seeing only one view
			\item nice addition, could be useful depending on what you want to do with the book
		\end{itemize}
\end{itemize}

\end{document}