\documentclass[main.tex]{subfiles}
\begin{document}

\subsubsection{Group Observations}

In the group prototype sessions we observed how group dynamics affected the direction of the discussion.Our first observation was that when a feature was presented and one of the participants had a strong negative opinion about it then the other participants that were unsure followed this opinion and it was impossible to change the group effect even if we started indicating some good points of this feature. \\

The next observation considers the speed with which the group arrives at a consensus depending on the starting idea. If this idea is perceived by the group as good, then everyone agreed and there was no further discussion, we had to doubt this idea to make the group start again the discussion.\\

The final observation regards how the group builds their perception of an idea on the first idea that comes up in the discussion. Even if the idea does not have a solid base, the other participants will add their ideas on the first one without refuting or doubting it. This is called group thinking~\cite{janis}.

\subsubsection{Individual Observations}
%Omar's observations
The first thing that was observed was that the individuals don't come up with new information after the first few points. Since they don't receive information or input from other people they are unable to create new pros and cons as they can only work with what they have in their own mind. \\
Another observation was noticeable once asking several individuals; since there is no communication between the individuals, they did not mention the same pros and cons. Since they don't communicate there is no influence on each other so their thought process stays there own; their thought processes won't `synchronize' and results in different ideas per person.

%CIndy
The individuals also asked questions if they were unsure. For example with the perspective: can it also swap?
\subsubsection{Analysis of the results}

The results are not analyzed quantativly as it is hard to judge if an individual opionion/view is unique compared to that of other individuals. There are sometimes slight differences in the way the opinions are stated, but the overall idea on the feature is the same. For both features the individuals found more results compared to the group. This was not expected from the theory. As explained in the Group observations section the group tended to follow one persons opinion and therefore overall gave less input. There were however cases were the group found errors that were not found by the individuals. During the discussion of the somewhat controversial feature of perspectives both groups came to the conclusion that this feature would be annoying for writers while none of the individuals came up with this concern. A possible reason behind this is the power of discussion, the group was able to discuss. Therefore the participants started thinking deeper about some features while the individuals received no extra input/stimulus after we stated the question which made it harder for them to think deeply about each of the features.
\end{document}