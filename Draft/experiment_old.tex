\section{Experiment}
We will try to change the mood of the session. Our idea is that a happy, positive group is less capable of critizing and more likely to rely on cognitive ease. 
They will approve or agree on features that would otherwise (in neutral or negative mood) considered less optimal. 

\paragraph{Experiment A}
We will do so by creating a highly enthousiast group by showing one group inspirational/achievement videos (``win compilations, satisfying gifs''). 

\paragraph{Experiment B}
One group will be put on their nerves by viewing ``less satisfying gifs''. These are gifs that typically end before showing the achievement, or have the achievement removed. The idea is that they are now more likely to look at errors and be skeptical to anything we could say. Thus, coming up with suggestions and additions we perhaps didn't think of and provide a useful prototyping session. 

\paragraph{Base group}
A last group shouldn't see any videos/gifs at all, to see if we actually varied our mood as intended.


\subsection{Possible outcomes}
We prefer critical thinking. Perhaps they even tear apart features (``Who needs that feature? Nobody wants to do X with an e-book") that would be considered ``ok'' or ``decent" in general. At least then we know our issues and usable features better (the features they agree on or like). 

A different scenario: In stead of creating a happy, cognitive ease group we get a highly motivated group. A group that is actively looking too improve our prototype. If the base group doesn't have this we could suggest everyone to use gifs, no matter if they are satisfying/un-satisfying.


Bad gifs:
http://imgur.com/a/ubCLK

Good gifs:
http://imgur.com/a/LprrK