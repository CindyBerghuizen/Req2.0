\documentclass{article}
\usepackage[utf8]{inputenc}
\usepackage[english]{babel}
\usepackage{hyperref, graphicx, verbatim, textcomp, booktabs, microtype, ntheorem}

\usepackage{fancyhdr}
\setlength{\headheight}{15.2pt}
\pagestyle{fancy}

\renewcommand{\arraystretch}{1.24}

\title{On effective prototype sessions for the e-book of the future}
\author{Cindy Berghuizen, Chiel Peters, Mary Marouli?, Omar 007, Fabi\"en Tesselaar}
\date{\today}

\begin{document}

\maketitle

\tableofcontents

\section{Introduction}

In this document we define our theory on prototyping (for e-books of the future).
First we define aspects can make a session typically bad or good. 
Then we provide an experiment setup to test our theory. Finally we provide 

\section{Bad prototyping}

Constantly harass into your own features
Provide distractions
Do not let system 2 work

Make a "draaiboek" for what to press and such (robot)
Manly only instructions
Don't pay attention to anything they say. (instructed robots)
Don't act upon their reactions.
Don't show anything outside the view of their own product.
Priming on good stuff
Put lots of people in one room
Continueing very long on a lot of ideas / features
non realistic assumptions / features

Give them some mind depleting exercises

Bad target group. (people who already like your product)

Projecting on what you know ( relate it on something they already love, iPhone6 answering the question if they
like the iPhone 5 / Apple)

priming, cognitive ease (situation they know)

Cognitive ease; making people comfortable by giving them food :)


Bad Session Theory:

- Bad Target Group -> Answering Easier Question (People who already like it)-> The Outside view, Chiel
- Cognitive ease -> Happy environment to prototyping (Ease,Mood and Intuition) Mary
- Deplete system 2, Cindy
- Anchoring to give yourself favorable results (LSN), Chiel
- Availability biasis. Mary
- Priming / Framing on the good features, CIndy

\subsection{Outcome}


\section{Good prototyping}

Good prototype session:
- Prototype session closely related to reality
- Intervene / let them experiment
- Priming creativity 

 
\subsection{Outcome}

Answers You Will Get
- Nonfuntional requirements *will they even buy it?
- They will give you new features you did not thought about
- They will find out what does not work.


\section{Experiment}

We will try to change the mood of the session. Our idea is that a happy, positive group is less capable of critizing and more likely to rely on cognitive ease. 
They will approve or agree on features that would otherwise (in neutral or negative mood) considered less optimal. 

\paragraph{Experiment A}
We will do so by creating a highly enthousiast group by showing one group inspirational/achievement videos (``win compilations, satisfying gifs''). 

\paragraph{Experiment B}
One group will be put on their nerves by viewing ``less satisfying gifs''. These are gifs that typically end before showing the achievement, or have the achievement removed. The idea is that they are now more likely to look at errors and be skeptical to anything we could say. Thus, coming up with suggestions and additions we perhaps didn't think of and provide a useful prototyping session. 

\paragraph{Base group}
A last group shouldn't see any videos/gifs at all, to see if we actually varied our mood as intended.


Bad gifs:
http://imgur.com/a/ubCLK

Good gifs:
http://imgur.com/a/LprrK

\end{document}