\documentclass{article}
\usepackage[utf8]{inputenc}
\usepackage[english]{babel}
\usepackage{hyperref, graphicx, verbatim, textcomp, booktabs, microtype, ntheorem}

\usepackage{fancyhdr}
\setlength{\headheight}{15.2pt}
\pagestyle{fancy}

\renewcommand{\arraystretch}{1.24}

\title{On effective prototype sessions for the e-book of the future}
\author{Cindy Berghuizen, Chiel Peters, Mary Gouseti,\\ Omar Pakker, Fabi\"en Tesselaar}
\date{\today}

\begin{document}

\maketitle

\tableofcontents

\section{Introduction}

In this document we define our theory on prototyping (for e-books of the future).
First we define aspects can make a session typically bad or good. 
Then we provide an experiment setup to test our theory. Finally we provide 

\section{Theory}

\subsection{Bad Target Group}
\subsection{Cognitive Ease}

\subsection{ Deplete System 2}
Ego depletion:\emph{ "if you have to force yourself to do something, yo are less willing
or less able to exert self-control when the next challenge comes around" }.

Ego depletion might be a way to let the subjects easier accept the prototype
as a good one. First let the subjects do something that takes a lot of effort, watch
a movie, try not to focus on something distracting etc. Then the prototype will
be shown. If the only parts shown of the system are the parts that are good
(as explained in 1.6) the subjects will be easier to except the system as good in
overall and do not criticize it.

\subsection{Anchoring}
\subsection{Availability}

\subsection{Priming / Framing on the good features / WYSIATI}
Only the good parts will be shown, the features that works the best and the easiest.
This is based on the WYSIATI principle. The only parts of the prototype
that are shown are the parts that work well or are really good. The parts of
the prototype that do not work well or are forgotten will be completely ignored.
Therefore the subjects will not notice the faults and limits of the system
and will believe the software works perfectly.


\section{Good prototyping}

Good prototype session:
- Prototype session closely related to reality
- Intervene / let them experiment
- Priming creativity 

 
\subsection{Outcome}
A good prototype session will provide you with usefull information. The subject
will experiment with the software / object, in the best case integrate it in their
daily life for a little while. While experimenting and using the device they will
find out what works well, what does not work, what they miss and what parts
are not used at all. This way they can provide better feedback and critique.
Also, a succesfull prototype will uncover the non-functional requirements.
For example: the subjects like the system but you dind out that they will not
buy it. Now work can be done to find out why people won't buy the system so you can make sure the system will be a success.


%Answers You Will Get
%- Nonfuntional requirements *will they even buy it?
%- They will give you new features you did not thought about
%- They will find out what does not work.


\section{Experiment - Anchoring}

In this expirement for the prototype session we will try to affect the results of the session by anchoring the participants. As described in the theory section anchoring (/ref{anchor}) the participants are required to fill out a form which states how much they like each of the new features (on a scale 1 to 10) that were mentioned in the requirements documents along which some more general questions on reading. On these question sheets the averages of a 'hypothetical' prior expirement will be stated as well. \\

To measure the anchoring effect we will hand out two different questionaires. The questions will remain the same however the averages we put after the questions will differ. One questionaire contains a high anchor 'average' and the other one a low anchor 'average'. Then after the experiment we will measure the difference between the high and the low one to see if the anchoring had any effect on our participants.\\

If anchoring effect is real and other factors such as small experiment group don't affect the results to much than the difference between the high anchor average and the low anchor average should be positive and we should be able to observe this for a number of questions in our questionaire.

\end{document}
