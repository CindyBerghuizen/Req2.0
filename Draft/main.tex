\documentclass{article}
\usepackage[utf8]{inputenc}
\usepackage[english]{babel}
\usepackage{hyperref, graphicx, verbatim, textcomp, booktabs, microtype, ntheorem}

\usepackage{fancyhdr}
\setlength{\headheight}{15.2pt}
\pagestyle{fancy}

\renewcommand{\arraystretch}{1.24}

\title{On effective prototype sessions for the e-book of the future}
\author{Cindy Berghuizen, Chiel Peters, Mary Gouseti,\\ Omar Pakker, Fabi\"en Tesselaar}
\date{\today}

\begin{document}

\maketitle

\tableofcontents

\section{Introduction}

In this document we define our theory on prototyping (for e-books of the future).
First we define aspects can make a session typically bad or good. 
Then we provide an experiment setup to test our theory. Finally we provide 

\section{Theory}

\subsection{Bad Target Group}
\subsection{Cognitive Ease \& Affect}
\emph{``When you are in a state of cognitive ease, you are probably in a good mood, like what you see, believe what you hear, trust your intuitions, and feel that the current situation is comfortably familiar.''}

\emph{``The affect heuristic is an instance of substitution, in which the answer to an easy question (How do I feel about it?) serves as an answer to a much harder question (What do I think about it?).''}

Setting an environment in which the subjects experience cognitive ease may lead them to be more receptive of the prototype. To accomplish that, the environment of the experiment will be comfortable and the subjects may be offered food or listen to music or even smile throughout the experiment. According to the first quote of this section, they will not be critical towards the prototype, they will accept what is suggested by the experiment and the prototype will seem familiar. Moreover, they will feel more comfortable using it than they would have. Finally, if they have to answer the questionnaire in the end of the experiment, they will have associated the memory of using the prototype with their good mood so they will be even more positive towards the prototype assuming they use the affect heuristic.

\subsection{ Deplete System 2}
Ego depletion:\emph{ "if you have to force yourself to do something, yo are less willing
or less able to exert self-control when the next challenge comes around" }.

Ego depletion might be a way to let the subjects easier accept the prototype
as a good one. First let the subjects do something that takes a lot of effort, watch
a movie, try not to focus on something distracting etc. Then the prototype will
be shown. If the only parts shown of the system are the parts that are good
(as explained in 1.6) the subjects will be easier to accept the system as good in
overall and do not criticize it.

\subsection{Anchoring}
\subsection{Availability}
\emph{``The availability heuristic, like other heuristics of judgment, substitutes one question for another: you wish to estimate the size of a category or the frequency of an event, but you report an impression of the ease with which instances come to mind.''}

Availability can be used to influence the results of the prototype session in a way that is desirable. For instance, the answer of a subject to the question whether she would use the prototype in her everyday life can be manipulated; if different situations that the subject could use an eTextbook, such as while traveling on a train, or when moving to a summer house, are mentioned before the question, the answer would be more positive. Similarly,  if situations that she could not use an eTextbook are mentioned before the question, the answer would be more negative. This would happen because the references before the question change the ease that examples of using eTextbooks in everyday life come to the mind of the subject.




\subsection{Priming / Framing on the good features / WYSIATI}
Only the good parts will be shown, the features that works the best and the easiest.
This is based on the WYSIATI principle. The only parts of the prototype
that are shown are the parts that work well or are really good. The parts of
the prototype that do not work well or are forgotten will be completely ignored.
Therefore the subjects will not notice the faults and limits of the system
and will believe the software works perfectly.


\section{Good prototyping}

Good prototype session:
- Prototype session closely related to reality
- Intervene / let them experiment
- Priming creativity 

 
\subsection{Outcome}
A good prototype session will provide you with usefull information. The subject
will experiment with the software / object, in the best case integrate it in their
daily life for a little while. While experimenting and using the device they will
find out what works well, what does not work, what they miss and what parts
are not used at all. This way they can provide better feedback and critique.
Also, a succesfull prototype will uncover the non-functional requirements.
For example: the subjects like the system but you dind out that they will not
buy it. Now work can be done to find out why people won't buy the system so you can make sure the system will be a success.


%Answers You Will Get
%- Nonfuntional requirements *will they even buy it?
%- They will give you new features you did not thought about
%- They will find out what does not work.


\section{Experiment}

We will try to change the mood of the session. Our idea is that a happy, positive group is less capable of critizing and more likely to rely on cognitive ease. 
They will approve or agree on features that would otherwise (in neutral or negative mood) considered less optimal. 

\paragraph{Experiment A}
We will do so by creating a highly enthousiast group by showing one group inspirational/achievement videos (``win compilations, satisfying gifs''). 

\paragraph{Experiment B}
One group will be put on their nerves by viewing ``less satisfying gifs''. These are gifs that typically end before showing the achievement, or have the achievement removed. The idea is that they are now more likely to look at errors and be skeptical to anything we could say. Thus, coming up with suggestions and additions we perhaps didn't think of and provide a useful prototyping session. 

\paragraph{Base group}
A last group shouldn't see any videos/gifs at all, to see if we actually varied our mood as intended.


Bad gifs:
http://imgur.com/a/ubCLK

Good gifs:
http://imgur.com/a/LprrK

\end{document}
