\documentclass{article}
\usepackage[utf8]{inputenc}
\usepackage[english]{babel}
\usepackage{hyperref, graphicx, verbatim, textcomp, booktabs, microtype, ntheorem}

\usepackage{fancyhdr}
\setlength{\headheight}{15.2pt}
\pagestyle{fancy}

\renewcommand{\arraystretch}{1.24}

\title{On effective prototype sessions for the e-book of the future}
\author{Cindy Berghuizen, Chiel Peters, Mary Marouli?, Omar 007, Fabi\"en Tesselaar}
\date{\today}

\begin{document}

\maketitle

\tableofcontents

\section{Introduction}

In this document we define our theory on prototyping (for e-books of the future).
First we define aspects can make a session typically bad or good. 
Then we provide an experiment setup to test our theory. Finally we provide 

\section{Bad prototyping}

Constantly harass into your own features
Provide distractions
Do not let system 2 work

Make a "draaiboek" for what to press and such (robot)
Manly only instructions
Don't pay attention to anything they say. (instructed robots)
Don't act upon their reactions.
Don't show anything outside the view of their own product.
Priming on good stuff
Put lots of people in one room
Continueing very long on a lot of ideas / features
non realistic assumptions / features

Give them some mind depleting exercises

Bad target group. (people who already like your product)

Projecting on what you know ( relate it on something they already love, iPhone6 answering the question if they
like the iPhone 5 / Apple)

priming, cognitive ease (situation they know)

Cognitive ease; making people comfortable by giving them food :)


Bad Session Theory:

- Bad Target Group -> Answering Easier Question (People who already like it)-> The Outside view, Chiel
- Cognitive ease -> Happy environment to prototyping (Ease,Mood and Intuition) Mary
- Deplete system 2, Cindy
- Anchoring to give yourself favorable results (LSN), Chiel
- Availability biasis. Mary
- Priming / Framing on the good features, CIndy

\subsection{Outcome}


\section{Good prototyping}

Good prototype session:
- Prototype session closely related to reality
- Intervene / let them experiment
- Priming creativity 

 
\subsection{Outcome}

Answers You Will Get
- Nonfuntional requirements *will they even buy it?
- They will give you new features you did not thought about
- They will find out what does not work.


\section{Experiment - Anchoring}

In this expirement for the prototype session we will try to affect the results of the session by anchoring the participants. As described in the theory section anchoring (/ref{anchor}) the participants are required to fill out a form which states how much they like each of the new features (on a scale 1 to 10) that were mentioned in the requirements documents along which some more general questions on reading. On these question sheets the averages of a 'hypothetical' prior expirement will be stated as well. \\

To measure the anchoring effect we will hand out two different questionaires. The questions will remain the same however the averages we put after the questions will differ. One questionaire contains a high anchor 'average' and the other one a low anchor 'average'. Then after the experiment we will measure the difference between the high and the low one to see if the anchoring had any effect on our participants.\\

If anchoring effect is real and other factors such as small experiment group don't affect the results to much than the difference between the high anchor average and the low anchor average should be positive and we should be able to observe this for a number of questions in our questionaire.

\end{document}