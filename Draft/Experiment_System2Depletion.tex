\documentclass[Main.tex]{subfiles} 
\begin{document}

\section{Experiment - System 2 Depletion}
As stated before:\emph{ "If you have to force yourself to do something, you are less willing or less able to exert self-control when the next challenge comes around" }.
This experiment will focus on this idea and attempt to influence the results.

\paragraph{Set-up}
For this experiment, 2 questionnaires will be created and 2 goups will each answer one of those. One of the questionnaires contains a large amount of boring and obvious questions before the questions we actually care about. The other questionnaire contains only the questions we want answers on. The idea behind this is that the group that has to answer a great amount of boring questions get annoyed and start to force themselves to go on with the questionnaire in such a way that they get it over with as soon as possible. In contrast, the other group has just the core questions and has therefore no need to force themselves through a long and boring questionnaire.

\paragraph{Possible Outcomes} %Expected Outcomes?
\begin{enumerate}
\item We expect the group that had to force themselves to finish the questionnaire to do it in a manner to get it done as fast as possible. Therefore we expect that they are more likely to answer for the first possible option. In contrast we expect the group that did not have to force themselves to finish the questionnaire properly and with their System 2 active and well therefore pick the first available answer possible less often in comparison.
\item The group that had to force themselves to finish the questionnaire answered the questions in a way to get done as fast as possible by picking random answers. In comparison with the other group the answers are more unbalanced; votes more evenly spread over every possible option instead of more votes for option X and/or Y as opposed to Z.

\end{document}
