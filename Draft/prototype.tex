\section{Prototype}
From all possible e-book features, the following we believe can be tested in our experiment: Perspectives, reading aids and interactive content.

\paragraph{Perspectives}

Users can select the perspective/role they want to read the e-book in. For instance, visualy or more text. It also allows teachers to save a perspective of ``the first eight chapters". During the prototyping session users would be able to read in two perspectives to see the differences.

\paragraph{Reading aids}
Users can look up definitions and synonyms of words by clicking on them. In addition, words and sentences could be translated or used in a different sentence for better understanding.

\paragraph{Interactive content}
It is al ready possible to add gifs and movies to pdf documents. This feature would add CDF (Computable Document Format) like features in the e-book. This way math functions and algorithms can be run interactively. An example can be included in the prototype to see if the content is now much easier to understand.
  